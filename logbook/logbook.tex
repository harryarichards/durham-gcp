\documentclass[12pt,a4paper]{article}
\usepackage{times}
\usepackage{durhampaper}
\usepackage{harvard}

\usepackage{algorithm, algpseudocode}

\usepackage{booktabs}
\usepackage{longtable}

\usepackage{graphicx}
\usepackage{wrapfig}

\usepackage{array}
\newcolumntype{P}[1]{>{\centering\arraybackslash}p{#1}}

\citationmode{abbr}
\bibliographystyle{agsm}



\begin{document}
\begin{tabular}[t]{p{.30\textwidth}  p{.55\textwidth}  p{.20\textwidth}}
  \textbf{Harry Alexander Richards}  & & \textbf{06/05/2019} \\
  \textbf{Daniel Paulusma} & & \textbf{Log Book} \\
  \textit{Durham University} & & 
\end{tabular}

\vspace{1cm}



\begin{longtable}{ p{.15\textwidth} | p{.90\textwidth} } 
\toprule
\hline 
\textbf{Date} & \textbf{Objectives} \\
\hline
\textbf{11/06/2018} & Discuss project with Daniel, set initial goals.  \\  \\
\textbf{15/09/2018} & Look over DIMACS graph structure, begin creating input tool.  \\  \\
\textbf{17/09/2018} & Complete input tool, research best graph representations.  \\  \\ 
\textbf{27/09/2018 - 25/10/2018}& Literature Review.  \\  \\  
\textbf{16/10/2018} & Implement adjacency list representation of graphs.  \\  \\  
\textbf{18/10/2018} & Implement the first-fit algorithm.  \\  \\  
\textbf{20/10/2018} & Implement vertex ordering heuristic - order by degree [smallest, largest and random].  \\  \\  
\textbf{06/11/2018} & Implement graph colouring verify.  \\  \\  
\textbf{10/11/2018} & Begin genetic algorithm, completed initialise population method.  \\  \\  
\textbf{12/11/2018} & Implement fitness method, measuring number of conflcits and $k$ value. Parent selection complete,  begin the crossover method but has issues.  \\  \\ 
\textbf{15/11/2018} & Changed graph repesentation from adjacency lists to networkx as its more convenient for implementation. \\  \\ 
\textbf{16/11/2018} & A simple crossover method implemented but not great in terms of reducing number of conflicts. \\  \\  
\textbf{17/11/2018} & Implemented a simple mutation method and iterate for fixed number of iterations, completed this simple genetic algorithm but not yet yielding proper colourings. \\  \\
\textbf{22/11/2018} & Alter approach completely, must now simply use the genetic algorithm to look for $k$-colourings and fitness algorithm evaluates the `number of conflicts' in each colouring. \\  \\
\textbf{05/12/2018} & Finished altering existing genetic algorithm so that it now works for the above approach. It produces proper colourings but there quality isn't great for alrge graphs - also struggles with large/dense graphs. \\  \\
\textbf{09/12/2018} & Alter current algorithms (greedy/first-fit and genetic), currently map vertices to a colour  but now we work with partitions of the set of vertices. \\  \\
\textbf{10/12/2018} & Altered both the initial population and the crossover method, now based aroudn those in Galinier and Hao's article. \textbf{Reminder: reference in paper}. \\  \\
\textbf{11/12/2018 - 11/01/2019} & Design report first draft. \\  \\
\textbf{17/01/2019 - 25/01/2019} & Received design report feedback, act on that and submit it. \\  \\
\textbf{25/01/2019} & Begin design report presentation. \\  \\
\textbf{28/01/2019} & Start simulated annealing, using the initial colouring method from genetic algorithm and use the mutation method to get solutions. \\  \\
\textbf{01/02/2019} & Give design report presentation, \\ \\
\textbf{03/02/2019} & Complete simulated annealing using temperature distribution defined. \\ \\
\textbf{05/02/2019} & Began final report outline. \\ \\
\textbf{12/02/2019} & Began GA-SA hybrid algorithm where a local search of simulated anneAling replaced the mutation operator. \\ \\
\textbf{17/02/2019} & Prepared formative demonstration for Daniel. \\ \\
\textbf{22/02/2019} & Completed the basic hybrid algorithm, doesn't work as well as expected. \\ \\
\textbf{24/02/2019} & Used DSATUR algorithm (by Brelaz), to gain an initial upper bound for the chromatic number. \\ \\
\textbf{26/02/2019} & Gave formative demonstration (later than expected due to illness). \\ \\
\textbf{03/03/2019} & Used matplotlib and networkx in order to visualise colourings, unfortunately the method we used seems to use colours that look very similar in some cases.  \\ \\
\textbf{09/03/2019} & Research tabu search, it is said to be superior to simulated annealing when hybridised with GA. \\ \\
\textbf{17/03/2019} & Began implementing tabu-GA hybrid, having issues with it producing empty sets. \\ \\
\textbf{25/03/2019} & Above issue resolved. \\ \\
\textbf{26/03/2019} & Began final report. \\ \\
\textbf{02/04/2019} & Acquiring final results in order to obtain graphs for report. \\ \\
\textbf{19/04/2019} & Draft report submitted. \\ \\
\textbf{29/04/2019} & Draft report feedback acquired - test implementations on more graphs. \\ \\
\textbf{01/05/2019} & All results acquired. \\ \\
\textbf{03/05/2019} & Submitted report. \\ \\
\hline
\bottomrule
\end{longtable}



\end{document}
